\documentclass[12pt,a4paper]{report}
\usepackage[utf8]{inputenc}
\usepackage[T1]{fontenc}
\usepackage{xcolor}
\usepackage[french]{babel}
\usepackage{amsmath}
%\usepackage{esint}

\title{ \Huge \textbf{Rapport du projet PlagueINT} \\ \large Modélisation de la propagation des épidémies}
%\date{\today}
\author{
CHERRE Romain
\and COROLLER Stevan 
\and PAMART Pierrick
\and PIPEREAU Yohan
\and \\
Encadrant: Mr. Vincent Gauthier }


\begin{document}
\maketitle

\tableofcontents

\newpage

\addcontentsline{toc}{section}{Modélisation du problème}
\section*{Modélisation du problème}
\begin{flushleft}
	On procède à un découpage de la carte du monde en celulles. Dans chaque celulles 
\end{flushleft}

\addcontentsline{toc}{subsection}{Présentation du modèle SIR} %obligé si on utilise pas les numérotations de section et sous section
\subsection*{Présentation du modèle SIR}

\begin{flushleft}
	Dans la suite, on note:
\begin{itemize}
	\item{N(t) le nombre d'individu total}
	\item{S(t) pour Susceptibles en anglais, le nombre d'individu qui n'ont pas encore été contaminés mais qui sont susceptible de l'être.}
	\item{I(t) pour Infectives en anglais, le nombre d'individu infectés par la maladie.}
	\item{R(t) pour Recovered en anglais, le nombre d'individu ayant été infectés, ayant survécu et étant immunisés.}
	\item{p, la probabilité qu'un individu infecté contamine un individu sain}
\end{itemize}
\end{flushleft}

\begin{flushleft}
	On obtient alors un jeu de trois équations différentielles ainsi qu'une équation de conservation de la population totale.
$$ N(t)=S(t)+I(t)+R(t) $$

	On peut également exprimer la variation du nombre d'individu sain par: $$ {dS(t) \over dt} = -p S I $$
où p est la probabilité qu'un individu infecté contamine un individu sain.
Ainsi, la quantité $$ pI $$ désigne la probabilité que les tous les infectés infectent un individu sain. Puis on multiplie par le nombre d'individu sain pouvant être infectés.

Par un raisonnement similaire 

\end{flushleft}

\begin{flushleft}
Les limites de notre modélisation sont les suivantes:
\end{flushleft}

\end{document}
