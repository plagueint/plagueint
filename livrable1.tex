\documentclass[12pt,a4paper]{report}
\usepackage[utf8]{inputenc}
\usepackage[T1]{fontenc}
\usepackage{xcolor}
\usepackage[french]{babel}
\usepackage{amsmath}
\usepackage{esint}
\title{Livrable 1 du projet PlagueINT sur la propagation des épidémies}
\date{16 février 2017}
\author{CHERRE Romain \\ COROLLER Stevan \\ PAMART Pierrick \\ PIPEREAU Yohan}
\begin{document}

\maketitle
\chapter*{Analyse des besoins}


\begin{flushleft}
Conseils:
-Le rapport peut être édité, à l'aide d'un traitement de textes choisi par
les membres du groupe-projet.
-Le rapport doit être soumis au format pdf (Portable Document Format).
-Le nom du fichier doit être explicite (par exemple : Rapport.pdf).
 -La page de garde doit indiquer le titre du projet, le nom des étudiants et de l'encadrant, la version courante et sa date.
-Les pages doivent être numérotées.
-Le rapport doit contenir une table des matières avec les numéros de pages.
-Des annexes peuvent si nécessaire être utilisées (par exemple, un lexique pour les définitions des abréviations éventuelles et/ou des termes techniques relatifs au sujet traité).
\end{flushleft}

\begin{flushleft}
Fonction du produit:
-Mode de visualisation (écoulement du temps) [discret ]
-Modélisation mondiale avec celulle de la taille d'un pays
-Possibilité d'exporter le résultat dans un fichier lisible [ lecture/écriture de fichier csv ]
-Voix de transports prise en compte [1 seul graphe avec les voix de transports sur les arêtes nbre de passagers/jour]
-Possibilité d'ajouter des événements (blocage d'aéroports, gare, etc..) au début
\end{flushleft}

\begin{flushleft}
Contraintes techniques:
-Utiliser Java8 + Eclipse + Python (pour récupérer les données et traiter ce qui est nécessaire)
-Possibilité d'exeution en mode terminal puis graphique
-Portabilité Windows, Linux (MAC OS) [natif avec Java]
\end{flushleft}

\begin{flushleft}
Critères d'acceptabilité et de réception
-Application performante [utiliser Euler dans un premier temps puis utiliser Runge Kutta implémenter par Java]
*********Interface utilisateur:*********
-temps [choix de l'échelle de temps pour gérer la rapidité de calcul]
-Choix des coeffs pour équa diff de manière directe (entrée manuelle) ou indirect (grippe->a=0.5)
-Le ou les Point de lancement de la maladie (Pays) [Rentrer pays et on regarde si ce qui est rentré match]
-Choisir Nombre d'infecté initial
\end{flushleft}

\begin{flushleft}
Extensions:
-Interface - graphique [spécifier la bibliothèque Java]
-Informations sur les celulles (petits graphiques, etc...) [utiliser des bibliothèques]
-Modification de l'environnement (hygiène, température, etc...) []
\end{flushleft}

\begin{flushleft}
Juridique: Creative Commons sans usage commerciale [BY NC SA]
\end{flushleft}

\tableofcontents
\end{document}

