\documentclass[12pt,a4paper]{report}
\usepackage[utf8]{inputenc}
\usepackage[T1]{fontenc}
\usepackage{xcolor}
\usepackage[french]{babel}
\usepackage{amsmath}
\usepackage{esint}

\title{Premier livrable de PlagueINT \\ \large Modélisation de la propagation des épidémies}
\date{\today}
\author{
CHERRE Romain
\and COROLLER Stevan 
\and PAMART Pierrick
\and PIPEREAU Yohan
\and \\
Encadrant: Mr. Christian Parrot }

\begin{document}
\maketitle

\tableofcontents

\addcontentsline{toc}{section}{Analyse des besoins}
\section*{Analyse des besoins}

\addcontentsline{toc}{subsection}{Fonction du produit} %obligé si on utilise pas les numérotations de section et sous section
\subsection*{Fonction du produit}
\begin{flushleft}
  \begin{itemize}
	\item[$\bullet$] Mode de visualisation (écoulement du temps) [discret ]
	\item[$\bullet$] Modélisation mondiale avec celulle de la taille d'un pays
	\item[$\bullet$] Possibilité d'exporter le résultat dans un fichier lisible [ lecture/écriture de fichier csv ]
	\item[$\bullet$] Voix de transports prise en compte [1 seul graphe avec les voix de transports sur les arêtes nbre de passagers/jour]
	\item[$\bullet$] Possibilité d'ajouter des événements (blocage d'aéroports, gare, etc..) au début
  \end{itemize}
\end{flushleft}

\addcontentsline{toc}{subsection}{Contraintes techniques}
\subsection*{Contraintes techniques}
\begin{flushleft}
  \begin{itemize}
	\item[$\bullet$] Utiliser Java8 + Eclipse + Python (pour récupérer les données et traiter ce qui est nécessaire)
	\item[$\bullet$] Possibilité d'exeution en mode terminal puis graphique
	\item[$\bullet$] Portabilité Windows, Linux (MAC OS) [natif avec Java]
    \end{itemize}
\end{flushleft}

\addcontentsline{toc}{subsection}{Critères d'acceptabilité et de réception}
\subsection*{Critères d'acceptabilité et de réception}
\begin{flushleft}
  \begin{itemize}
	\item[$\bullet$] Application performante [utiliser Euler dans un premier temps puis utiliser Runge Kutta implémenter par Java]
*********Interface utilisateur:*********
	\item[$\bullet$] temps [choix de l'échelle de temps pour gérer la rapidité de calcul]
	\item[$\bullet$] Choix des coeffs pour équa diff de manière directe (entrée manuelle) ou indirect (grippe->a=0.5)
	\item[$\bullet$] Le ou les Point de lancement de la maladie (Pays) [Rentrer pays et on regarde si ce qui est rentré match]
	\item[$\bullet$] Choisir Nombre d'infecté initial
  \end{itemize}
\end{flushleft}

\addcontentsline{toc}{subsection}{Extensions}
\subsection*{Extensions}
\begin{flushleft}
  \begin{itemize}
	\item[$\bullet$] Interface - graphique [spécifier la bibliothèque Java]
	\item[$\bullet$] Informations sur les celulles (petits graphiques, etc...) [utiliser des bibliothèques]
	\item[$\bullet$] Modification de l'environnement (hygiène, température, etc...) []
  \end{itemize}
\end{flushleft}

\addcontentsline{toc}{subsection}{Juridique}
\subsection*{Juridique}
\begin{flushleft}
Creative Commons sans usage commerciale [BY NC SA]
\end{flushleft}

\addcontentsline{toc}{section}{Spécification fonctionnelle générale}
\section*{Spécification fonctionnelle générale}

\addcontentsline{toc}{section}{Regroupement modulaire des fonctionnalités}
\section*{Regroupement modulaire des fonctionnalités}
\begin{flushleft}
	\begin{itemize}
		\item[$\bullet$] Visualisation
			\begin{itemize}
				\item[$\bullet$] Terminal
				\item[$\bullet$] Graphique
				\item[$\bullet$] Exportation en CSV
			\end{itemize}
		\item[$\bullet$] Évènements
			\begin{itemize}
				\item[$\bullet$] Blocage de lieux de transports
				\item[$\bullet$] Blocage des frontières
			\end{itemize}
		\item[$\bullet$] Statistiques
			\begin{itemize}
				\item[$\bullet$] Par pays : évolutions du nombre d'infectés, ...
				\item[$\bullet$] Générales
			\end{itemize}
		\item[$\bullet$] Calcul des évolutions temporelles
	\end{itemize}
\end{flushleft}

\addcontentsline{toc}{section}{Description du flux des données entre les modules}
\section*{Description du flux des données entre les modules}

\end{document}
